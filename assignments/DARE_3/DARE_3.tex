

\documentclass[a4paper, 11pt]{article} 



\usepackage{graphicx} 

\usepackage[parfill]{parskip} % Activate to begin paragraphs with an empty line rather than an indent

%%% PACKAGES
\usepackage{booktabs} % for much better looking tables
\usepackage{array} % for better arrays (eg matrices) in maths
\usepackage{paralist} % very flexible & customisable lists (eg. enumerate/itemize, etc.)
\usepackage{verbatim} % adds environment for commenting out blocks of text & for better verbatim
\usepackage{subfig} % make it possible to include more than one captioned figure/table in a single float

\renewcommand{\thesection}{\Alph{section}.}

\title{EDLD 650: Data Analysis and Replication Exercise (DARE) 3}
\author{David D. Liebowitz}
\date{Due: Feb. 18, 2024} 

\begin{document}
\maketitle

In this project, you will replicate and extend analysis from Kim, Capotosto, Hartry and Fitzgerald (2011). In the dataset EDLD\_650\_DARE\_3.csv, you will find the variables listed in Table 1. 

\begin{table}[!htbp] \centering 
  \caption{Variable definitions} 
  \label{} 
\begin{tabular}{ll} 
\\[-1.8ex]\hline 
\\[-1.8ex] Variable name & Description \\ 
\hline \\[-1.8ex]
id & Unique student identifier \\
school & School of attendance (1-4) \\
dorf & Baseline DIBELS Oral Reading Fluency score \\
frpl & Receives free- or reduced-price lunch \\
female & Female (1=yes) \\
treat & Student assigned to after-school READ180 intervention (1=yes) \\
read180\_attend & Proportion of days attending READ180 (in 7 month window) \\
sat10\_compreh & Post-intervention test of comprehension \\

\hline
\hline

\end{tabular}
\end{table}

\section{Baseline randomization checks (1 point)}
For the following tasks, give your best attempt at completing the analysis and write-up. If you are unable to conduct the programming or analysis, describe what you are attempting to do and what your results would mean.

\begin{enumerate}
	\item[A1.] Create a table comparing the baseline characteristics (family income, gender, test score) for students assigned to the treatment and control conditions. Assess and describe whether the randomization process generated identical treatment and control conditions. Describe the results of your assessment in 1-2 sentences. If it did not (or if it had not), would this invalidate the causal claims of the study? Why or why not?

\end{enumerate}

\section{Replication and Extension (9 points)}
For the following tasks, give your best attempt at completing the analysis and write-up. If you are unable to conduct the programming or analysis, describe what you are attempting to do and what your results would mean. 

\begin{enumerate}
	\item[B1.] Estimate the bivariate relationship between students' final reading comprehension outcomes and their attendance rate (proportion of days attended) in a seven-month READ180 program. Present these results in a table with an accompanying discussion of what these results show and whether they should be understood as the causal effect of READ180 on reading comprehension outcomes in 1 paragraph.
	\item[B2.] Compare the average post-test reading comprehension scores of students who were assigned to participate in the READ180 intervention with those who were not. Present a figure comparing these mean differences. Is the difference in these scores meaningful and does the difference reflect anything other than sampling idiosyncrasy? 
	\item[B3.] Estimate Intent-to-Treat estimates of being assigned to participate in an after-school READ180 intervention. Present these results in a table and an accompanying write-up as you would report these in an academic paper in 1 paragraph. What differences are there in the results you estimated in response to this question and those for question B2?
	\item[B4.] Identify the effects of full participation in a seven month after-school READ180 reading intervention. In other words: what are the effects of 100 percent attendance in a seven-month reading program, compared to not attending at all? Describe the model you estimate, its accompanying assumptions and defend the extent to which these assumptions are met in your analysis. Present these results in a table and an accompanying write-up as you would report these in an academic paper in 2-3 paragraphs.
	\item[B5.] Write a discussion paragraph in which you present the substantive conclusions (and limitations) of your results about the effects of the after-school READ180 intervention you have documented.
	
\end{enumerate}

\end{document}
