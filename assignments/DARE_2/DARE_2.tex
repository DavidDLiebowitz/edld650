

\documentclass[a4paper, 11pt]{article} 



\usepackage{graphicx} 

\usepackage[parfill]{parskip} % Activate to begin paragraphs with an empty line rather than an indent

%%% PACKAGES
\usepackage{booktabs} % for much better looking tables
\usepackage{array} % for better arrays (eg matrices) in maths
\usepackage{paralist} % very flexible & customisable lists (eg. enumerate/itemize, etc.)
\usepackage{verbatim} % adds environment for commenting out blocks of text & for better verbatim
\usepackage{subfig} % make it possible to include more than one captioned figure/table in a single float

\renewcommand{\thesection}{\Alph{section}.}

\title{EDLD 650: Data Analysis and Replication Exercise (DARE) 2}
\author{David D. Liebowitz}
\date{Due: Feb. 4, 2024} 

\begin{document}
\maketitle

In this project, you will replicate and extend analysis from Holden (2016). In the dataset EDLD\_650\_CA\_school\_es.dta, you will find the variables listed in Table 1. The dataset includes school years 2003-2009. Without binning the forcing variable, your figures will look slightly different than those in Holden. This is fine and will not affect the substantive conclusions. If you would like to set yourself a data management challenge, attempt to mirror the figures exactly! Pay careful attention throughout to which years you should be using in your analyses.

\begin{table}[!htbp] \centering 
  \caption{Variable definitions} 
  \label{} 
\begin{tabular}{ll} 
\\[-1.8ex]\hline 
\\[-1.8ex] Variable name & Description \\ 
\hline \\[-1.8ex]
cds & Unique school id \\
sname & School name \\
distid & District id \\
year & School year (spring) \\
readingscore & Mean reading scaled score \\
mathscore & Mean math scaled score \\
api\_rank & Academic performance rank in 2003;  \\
 & determines eligibility for Williams \\
yrs\_teach & Average years of teaching experience \\
yrs\_dist & Average years of teaching experience in district \\
pct\_ai & Percent of American Indian students \\
pct\_as & Percent of Asian students \\
pct\_pi & Percent of Pacific Islander students \\
pct\_fi & Percent Filipino students  \\
pct\_hi & Percent Hispanic students \\
pct\_aa & Percent African American students \\
pct\_wh & Percent White students \\
pct\_other & Percent students w/ missing or mult responses \\
percentfrl & Percent free- and reduced-price lunch \\
total & Total school enrollment \\
classsize & Average class size \\
average\_score & Avg. standardized math/reading score by school \\
api\_mean & Avg. standardized math/reading score by API rank \\
norm & Centered API score \\
ind & Intended recipient of Williams based on 2003 API score (0/1) \\
ind\_norm & Interaction of centered API (norm) and intent to treat (ind) \\
receive\_williams & Actual recipient of Williams in 2005 (0/1, NA if year!=2005) \\
\hline
\hline

\end{tabular}
\end{table}

\section{Assumption tests (4 points)}
For the following tasks, give your best attempt at completing the analysis and write-up. If you are unable to conduct the programming or analysis, describe what you are attempting to do and what your results would mean.

\begin{enumerate}
	\item[A1.] Create a figure that describes whether the forcing variable predicts the question variable of interest. Specifically, did a school with an Academic Performance Index (API) of 643 or lower \textbf{\textit{in 2003}} receive additional instructional material funding \textbf{\textit{in 2005}}? Present the figure and associated write-up as you would report these in an academic paper in 2-3 sentences.
	\item[A2.] Is there evidence of schools manipulating their placement around the discontinuity? Present at least two figures demonstrating (a) whether there is evidence that schools attempted to receive an API score that would have made them eligible to receive additional funding; and (b) whether there is evidence that schools that did and did not receive Williams funding were different in observable ways. These figures should present characteristics of schools that are exogenous to the receipt of Williams funding (\textit{think about what would be exogenous in this case}). Present the figures and associated write-up as you would report these in an academic paper in 1-2 paragraphs.
	\item[A3.] \textbf{\textit{Optional Extension}} Construct a table of summary statistics. What are the sample characteristics of the subset of elementary schools in our dataset? What is different about our data from the full data used in Holden (2016) and how might that affect the interpretation of results? Present the table and associated write-up as you would report these in an academic paper in 1-2 paragraphs.

\end{enumerate}

\section{Replication and Extension 6 points)}
For the following tasks, give your best attempt at completing the analysis and write-up. If you are unable to conduct the programming or analysis, describe what you are attempting to do and what your results would mean. For these tasks, think about what bandwidth is the correct one to present for your main results.

\begin{enumerate}
	\item[B1.] Did the receipt of additional funds for instructional materials improve test score outcomes for elementary students in California? Construct a figure that presents graphical evidence in support of your answer to this question. Pay close attention to the bandwidth of analysis Holden selects. Either use the same bandwidth or justify a different selection. Write up your description of this figure in one (1) paragraph as you would for an academic paper. If you have not completed the previously mentioned data management tasks to bin the forcing variable, explain how and why your figure looks different from the original figure in 2-3 sentences.
	\item[B2.] In a regression framework, formally test whether the receipt of additional funds for instructional materials improved test score outcomes for elementary students in California. Present these results in a table and associated 1-2 paragraph write-up as you would in an academic paper.
	\item[B3.] Write a discussion paragraph in which you present the substantive conclusions (and limitations) of your results about the effects of added textbook funding for California elementary school students.
	\item[B4.] \textbf{\textit{Optional Extension}} Present a series of robustness checks to the main results you have found. Consider varying year, functional form and bandwidth of your estimates.
\end{enumerate}

\end{document}
